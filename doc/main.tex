\documentclass[12pt]{article}

\usepackage{subcaption}
\usepackage{underscore}
\usepackage{rka-style}
\usepackage{multirow}
\usepackage[margin=0.875in]{geometry}
\usepackage{accents, hyperref, chngcntr, tikz-cd}
\usepackage[linesnumbered, noline]{algorithm2e}
\usepackage{natbib}
\bibliographystyle{plainnat}
\setcitestyle{authoryear,round,citesep={;},aysep={,},yysep={;}}

\hypersetup{ %
    pdfborder=0 0 0,
    pdfpagemode=UseNone,
    colorlinks=true,
    linkcolor=blue,
    citecolor=blue,
    filecolor=blue,
    urlcolor=blue,
    pdfview=FitH}

\graphicspath{ {figures/} }
\usepackage{cleveref,booktabs}
\usepackage[parfill]{parskip}

% TO DO NOTES
\usepackage[colorinlistoftodos,bordercolor=orange,backgroundcolor=orange!20,linecolor=orange,textsize=scriptsize]{todonotes}
\newcommand{\sra}[1]{\todo[inline]{{\textbf{Sara:} \emph{#1}}}}
\newcommand{\ahmed}[1]{\todo[inline]{{\textbf{Ahmed:} \emph{#1}}}}

% Preamble End
% Document Start

\begin{document}

\title{Big Data Project Report}
\author{Ahmed Essam, Ahmed Khaled, Sara Maher, Ibrahim Mahmoud}

\maketitle

\abstract{This report gives a time series analysis of historical Walmart sales data about more than three thousand different products sold across three different states in the United States. The data is part of the M5 forecasting competition \citep{Makridakis2020} and in addition to the time series analysis we also propose and describe a forecasting system and analyze its prediction capabilities.}

\tableofcontents
\clearpage
\setlength{\parindent}{0pt}
\setlength{\parskip}{\baselineskip}
\section{Introduction}
Sales forecasting is an important application. Especially for business as it enables companies to make better, more informed decisions. This is best done with enough information about previous sales at hand, and using it to predict future revenue and growth.

\subsection{Problem Definition}
We are given Walmart sales data from ten stores in three different states. Our project can be divided into two main problems: the first is to answer business-related questions on the dataset (data analysis) and the second is to devise a machine learning model that can predict the sales of each item offered by the stores twenty eight days into the future. 

This document is organized as follows: we review the dataset next, perform exploratory data analysis in the succeeding section, and outline our efforts at developing a forecasting model (and its evaluation) in the last section.

\subsection{Dataset Overview}
The dataset is given by the \emph{M5 competition} available on Kaggle\footnote{\url{https://www.kaggle.com/c/m5-forecasting-accuracy}.} which is the current installment in the popular M competition series on forecasting \citep{Makridakis2020}. The dataset includes time series data of the sales of various Walmart store products divided hierarchically by the item level, department, product category, and geographical area. The dataset also includes explanatory variables such as price, promotions, day of the week, and special events (e.g. Valentine's Day, Orthodox Easter, and the Super Bowl, one of the largest sporting events in American Football). There are $3,075$ products classified in $3$ product categories and $7$ product departments. The products are sold across $10$ stores in $3$ different states (California, Texas, and Wisconsin). The total number of M5 series across the entire hierarchy is $42,840$.  The dataset guide is given on the \href{https://mk0mcompetitiont8ake.kinstacdn.com/wp-content/uploads/2020/02/M5-Competitors-Guide_Final-1.pdf}{on the M5 competition website}.

\clearpage
\section{Data Analysis}
\subsection{Dataset Files}
The dataset is divided into four files, each of which is described in \Cref{tab:dataset-desc}.

\begin{table}[h]
    \centering
    \caption{Dataset description}
    \label{tab:dataset-desc}
    \begin{tabular}{@{}ccc@{}}
    \toprule
    File name                  & Description                                                                                                & Shape               \\ \midrule
    sales_train_validation.csv & Contains the sales data for each item.                                                                      & $30490 \times 1919$ \\
    calendar.csv & \begin{tabular}[c]{@{}c@{}}Contains calendar dates as well as \\ any events that happen on each date.\end{tabular} & $1969 \times 14$ \\
    sell_prices.csv            & \begin{tabular}[c]{@{}c@{}}Contains the sell price for each item\\ divided by store and week.\end{tabular} & $6841121 \times 4$  \\
    sample_submission.csv      & \begin{tabular}[c]{@{}c@{}}Contains sample submission forecasting\\ for the competition.\end{tabular}      & $60980 \times 29$   \\ \bottomrule
    \end{tabular}
\end{table}

The columns in each file are described in \Cref{tab:calender-desc}, \Cref{tab:sales-desc} and \Cref{tab:prices-desc}.
The columns in each file are described in \Cref{tab:calendar-desc}.

% Please add the following required packages to your document preamble:
% \usepackage{booktabs}
\begin{table}[]
    \centering
    \caption{calendar.csv columns description. Note that SNAP stands for 
    Supplemental Nutrition Assistance Program, a program to help provide food for low-income families in the United States of America.}
    \label{tab:calendar-desc}
    \begin{tabular}{@{}cc@{}}
    \toprule
    Column Name                                                       & Description                         \\ \midrule
    Date                                                              & The full date as a string.          \\
    wm_yr_wk                                                          & Week id.                            \\
    \begin{tabular}[c]{@{}c@{}}weekday/wday\\ month/year\end{tabular} & Date split up.                      \\
    event_name_1/2                                                    & Name of this event (if applicable). \\
    event_type_1/2                                                    & Type of the event (if applicable).  \\
    \begin{tabular}[c]{@{}c@{}}snap_CA\\ snap_TX\\ snap_WI\end{tabular} & \begin{tabular}[c]{@{}c@{}}Indicator of whether the stores allow\\ SNAP purchases.\end{tabular} \\ \bottomrule
    \end{tabular}
\end{table}



\section{Exploratory Data Analysis}
In this section, we analyze and visualize the data from 3 files; calendar.csv, \\ sales_train_validation.csv, and sell_prices.csv.

\subsection{Calender-Focused Analysis}
The data starts at \date{January 29, 2011} to \date{June 19, 2016}, with day-count up to 1969. The dataset contains 30 events, some of them are repeated annually(counting duplicates, there are 167 events). They are divided into: Cultural, National, Religious and Sporting events. With Religious events being the most frequent with 55 occurrence. We also have 10 SNAP days each month in each state. Columns descriptions are in \Cref{tab:calender-desc}.
\begin{table}[H]
    \centering
    \caption{Calender details}
    \label{tab:calender-desc1}
    \begin{tabular}{@{}lllll@{}}
    \toprule
    File            & Column name                                                                                                  & Column description &  &  \\ \midrule
    \multirow{7}{*}{calender.csv}    & 	\texttt{date} & The date in Year-Month-Day format. \\ 
                    & \texttt{wm\_yr\_wk}                                                                         & \begin{tabular}[c]{@{}l@{}}Code the week year from the dataset\\ starting date.\end{tabular}              \\
                    & \texttt{weekday}                                                                            & The weekday.                                                                                             \\
                    & \texttt{wday}                                                                               & \begin{tabular}[c]{@{}l@{}}The weekday's number, starting from\\ Saturday.\end{tabular}                   \\
                    & \texttt{month}                                                                              & The month.                                                                                                \\
                    & \texttt{year}                                                                               & The year.                                                                                                 \\
                    & \texttt{d}                                                                                  & \begin{tabular}[c]{@{}l@{}}The incremental ID of the day in the\\ dataset.\end{tabular}                   \\
                    & \texttt{event\_name\_1}                                                                     & \begin{tabular}[c]{@{}l@{}}Name of the event occurring in this day,\\ if one exists.\end{tabular}         \\
                    & \texttt{event\_type\_1}                                                                     & \begin{tabular}[c]{@{}l@{}}Type of the event occurring in this day,\\ if one exists.\end{tabular}         \\
                    & \texttt{event\_name\_2}                                                                     & \begin{tabular}[c]{@{}l@{}}Name of the second event occurring in\\ that date,if one exists.\end{tabular}  \\
                    & \texttt{event\_type\_2}                                                                     & \begin{tabular}[c]{@{}l@{}}Type of the second event occurring in\\ that date, if one exists.\end{tabular} \\
                    & \texttt{snap\_CA}, \texttt{snap\_TX}, \texttt{snap\_WI} & \begin{tabular}[c]{@{}l@{}}Equals 1 if SNAP\footnote{The Supplement Nutrition      Assistance Program (SNAP) provides low income families and individuals with an Electronic Benefits Transfer      debit card to purchase food products, this purchasing process is done monthly across 10 days} purchases are allowed on this \\ date.\end{tabular}  \\ \bottomrule
    \end{tabular}
    \end{table}

\subsection{Sales-Focused Analysis}
The sales data covers $1913$ days only, from \textit{\date{January 29, 2011}} to \textit{\date{April 24, 2016}}, unlike the calender which covered $1969$ days. The data contains $30490$ rows and $1919$ columns, with sales data from 3 states; California (CA), Texas (TX) and Wisconsin (WI), with $4$, $3$, and $3$ stores in each respectively. It also contains $4$ categories of products; Hobbies, Household, Foods, with $2$, $2$, and $3$ departments respectively. Columns descriptions are in \Cref{tab:sales-desc}.

    \begin{table}[H]
        \centering
        \caption{Sales details}
        \label{tab:sales-desc}
        \begin{tabular}{@{}lllll@{}}
        \toprule
        File            & Column name                                                                                                  & Column description &  &  \\ \midrule
    \multirow{7}{*}{sales\_train\_validation.csv} & \texttt{id}                                                                                            & \begin{tabular}[c]{@{}l@{}}With the id of the item codes as \\ \textit{item\_id}\_\textit{store\_id}\_validation.\end{tabular}  \\
                                                       & \texttt{item\_id}                                                                                      & Product ID.                                                                                                           \\
                                                       & \texttt{dept\_id}                                                                                      & Department ID.                                                                                                        \\
                                                       & \texttt{cat\_id}                                                                                       & Category ID.                                                                                                          \\
                                                       & \texttt{store\_id}                                                                                     & Store ID.                                                                                                             \\
                                                       & \texttt{state\_id}                                                                                     & State ID.                                                                                                             \\
                                                       & \texttt{d\_1}, \texttt{d\_2}, \dots, \texttt{d\_1913} & \begin{tabular}[c]{@{}l@{}}$1913$ columns, containing the total \\ sold items each day.\end{tabular}
        \\ \bottomrule
        \end{tabular}
        \end{table}

\subsubsection{Sales Exploration Figures}
    
    \textbf{How is the product distribution?}\\
    Products are distributed equally on the different stores by Walmart, without favoring any of them, as shown in \Cref{fig:SE1}.
    \begin{figure}[H]
    \centering
        \includegraphics[width=0.5\textwidth]{{"Share of each store from the products"}}
    \caption{Share of each store from the products}
    \label{fig:SE1}
    \end{figure}

    \textbf{How good is this distribution scheme?}\\
    To know that, we need to check the total sales and revenue. So we add a total sales column to the data. From \Cref{fig:SE2}, we find that no, not all states are equally profitable, California provides $1.5$ of the sales of each of Texas and Wisconsin. CA\_3 is especially profitable, amounting to $17\%$ of the total sales.
    \begin{figure}[H]
        \begin{subfigure}{.5\textwidth}
          \centering
          \includegraphics[width=.8\linewidth]{{"Sales percentage of each state"}}  
          \caption{Sales percentage of each state}
        \end{subfigure}
        \begin{subfigure}{.5\textwidth}
          \centering
          \includegraphics[width=.8\linewidth]{{"Sales percentage of each store"}}  
          \caption{Sales percentage of each store}
        \end{subfigure}
        \caption{Sales percentage for the states and stores}
        \label{fig:SE2}
    \end{figure}

    \textbf{What is the most popular category?} \\
    We found that FOODS is the most popular category, with a staggering $68.6\%$ of the total sales, and the most popular department is FOODS_3, with $49.3\%$ of the sales. While the least popular category and department is HOBBIES and HOBBIES_2, with $9.3\%$ and $0.8\%$ of the total sales respectively. This is shown in \Cref{fig:SE3}.
    \begin{figure}[H]
        \begin{subfigure}{.5\textwidth}
          \centering
          \includegraphics[width=.8\linewidth]{{"Sales percentage of each category"}}  
          \caption{Sales percentage of each category}
        \end{subfigure}
          \begin{subfigure}{.5\textwidth}
            \centering
            \includegraphics[width=.8\linewidth]{{"Sales percentage of each department"}}  
            \caption{Sales percentage of each department}
        \end{subfigure}
        \begin{subfigure}{.5\textwidth}
            \centering
            \includegraphics[width=.8\linewidth]{{"Total Sales per State for Each Category"}}  
            \caption{Total Sales per State for Each Category}
          \end{subfigure}
          \begin{subfigure}{.5\textwidth}
              \centering
              \includegraphics[width=.8\linewidth]{{"Total Sales per State for Each Department"}}  
              \caption{Total Sales per State for Each Department}
            \end{subfigure}
            \begin{subfigure}{\textwidth}
                \centering
                \includegraphics[width=.5\linewidth]{{"Percentage of Each Department Sales for Each Store"}}  
                \caption{Percentage of Each Department Sales for Each Store}
            \end{subfigure}
        \caption{Total sales for the categories and departments}
        \label{fig:SE3}
    \end{figure}

    \textbf{Is the current stocking scheme good?}\\
    Yes, Walmart stocks up on a variety of products of the most popular category and department very well, as shown in \Cref{fig:SE4}.
    \begin{figure}[H]
        \begin{subfigure}{.5\textwidth}
          \centering
          \includegraphics[width=.8\linewidth]{{"Share of each category from the products"}}  
          \caption{Share of each category from the products}
        \end{subfigure}
        \begin{subfigure}{.5\textwidth}
          \centering
          \includegraphics[width=.8\linewidth]{{"Share of each department from the products"}}  
          \caption{Share of each department from the products}
        \end{subfigure}
        \caption{Share of each category and department from the products}
        \label{fig:SE4}
    \end{figure}

    \textbf{How do sales fluctuate over time?} \\
    Generally, they increase throughout the covered period as shown in \Cref{fig:SE5}, especially from $2011$ to $2012$ as shown in \Cref{fig:SE7}. The observed sales data has a trend and it is highly seasonal, as shown in \Cref{fig:SE6}.
    \begin{figure}[H]
        \begin{subfigure}{.5\textwidth}
          \centering
          \includegraphics[width=.8\linewidth]{{"Total sales over time"}}  
          \caption{Total sales over time}
          \label{fig:SE5}
        \end{subfigure}
        \begin{subfigure}{.5\textwidth}
            \centering
            \includegraphics[width=.8\linewidth]{{"Sales in different years"}}  
            \caption{Sales in different years}
            \label{fig:SE7}
          \end{subfigure}
        \begin{subfigure}{\textwidth}
          \centering
          \includegraphics[width=.5\linewidth]{{"Time Series Decomposition"}}  
          \caption{Time Series Decomposition}
          \label{fig:SE6}
        \end{subfigure}
        \caption{Sales as a time series}
    \end{figure}

   \textbf{How do events affect sales?} \\
   By zooming in on one year specifically, from \date{January 29, 2011} to \date{January 29, 2012}, shown in \Cref{fig:SE8}, we find that, surprisingly, sales don't go through a substantial spike around most events. Still we find that from all the events categories, days of religious events\footnote{Except for Christmas, on which the sales are zero, and these Walmart stores probably close} have the highest sales, as shown in \Cref{fig:SE9}, while the highest sales day given an event is on the SuperBowl day, as shown in \Cref{fig:SE10}.
   \begin{figure}[H]
    \begin{subfigure}{.5\textwidth}
      \centering
      \includegraphics[width=.8\linewidth]{{"2011 Q1"}}  
      \caption{First Quarter}
    \end{subfigure}
    \begin{subfigure}{.5\textwidth}
        \centering
        \includegraphics[width=.8\linewidth]{{"2011 Q2"}}  
        \caption{Second Quarter}
      \end{subfigure}
    \begin{subfigure}{.5\textwidth}
        \centering
        \includegraphics[width=.8\linewidth]{{"2011 Q3"}}  
        \caption{Third Quarter}
    \end{subfigure}
    \begin{subfigure}{.5\textwidth}
        \centering
        \includegraphics[width=.8\linewidth]{{"2011 Q4"}}  
        \caption{Fourth Quarter}
    \end{subfigure}
    \caption{Events occurring in 2011}
    \label{fig:SE8}
\end{figure}
\begin{figure}[H]
    \begin{subfigure}{.5\textwidth}
      \centering
      \includegraphics[width=.8\linewidth]{{"Total Sales for Each Event"}}  
      \caption{Total Sales for Each Event}
      \label{fig:SE10}
    \end{subfigure}
    \begin{subfigure}{.5\textwidth}
        \centering
        \includegraphics[width=.8\linewidth]{{"Total Sales for Each Event Category"}}  
        \caption{Total Sales for Each Event Category}
        \label{fig:SE9}
      \end{subfigure}
    \caption{Total Sales given an event}
\end{figure}

\textbf{When are the highest sales?} \\
As shown in \Cref{fig:SE11}, March is the highest total sales month, which is surprising as it comes second, after February, according to the number on events happening in it. The sales are high in the first 15 days of the month too, which makes sense as it is the time where people are more prone to buy things, after cashing the monthly paycheck, and most of the sales happen on the weekend.
\begin{figure}[H]
    \begin{subfigure}{.3\textwidth}
      \centering
      \includegraphics[width=.8\linewidth]{{"Total Sales per month"}}  
      \caption{Total Sales per month}
    \end{subfigure}
    \begin{subfigure}{.3\textwidth}
        \centering
        \includegraphics[width=.8\linewidth]{{"Total Sales per day in the month"}}  
        \caption{Total Sales per day in the month}
      \end{subfigure}
    \begin{subfigure}{.3\textwidth}
        \centering
        \includegraphics[width=.8\linewidth]{{"Total Sales for each day of the week"}}  
        \caption{Total Sales for each day of the week}
      \end{subfigure}
    \caption{Total Sales given a certain month, day, or weekday}
    \label{fig:SE11}
\end{figure}

Zooming in on states, we find that the absolute month with the maximum sales for each state:
\begin{itemize}
    \item California: \date{August, 2015}.
    \item Texas: \date{August, 2013}.
    \item Wisconsin: \date{March, 2016}.
\end{itemize}
With California showing the maximum increase over the years. The month with the maximum sales in each state was the same as the one with the maximum total sales, March. This is shown in \Cref{fig:SE12}. This is also the case for departments, shown in \Cref{fig:SE13}.

\begin{figure}[H]
    \begin{subfigure}{.5\textwidth}
      \centering
      \includegraphics[width=.8\linewidth]{{"Monthly sales in each state"}}  
      \caption{Monthly sales in each state}
    \end{subfigure}
    \begin{subfigure}{.5\textwidth}
        \centering
        \includegraphics[width=.8\linewidth]{{"Per Month sales in each state"}}  
        \caption{Per Month sales in each state}
      \end{subfigure}
    \caption{Total monthly sales given a state}
    \label{fig:SE12}
\end{figure}

\begin{figure}[H]
      \begin{subfigure}{.5\textwidth}
        \centering
        \includegraphics[width=.8\linewidth]{{"Monthly sales for each department"}}  
        \caption{Monthly sales for each department}
      \end{subfigure}
      \begin{subfigure}{.5\textwidth}
        \centering
        \includegraphics[width=.8\linewidth]{{"Per Month sales for each department"}}  
        \caption{Per Month sales for each department}
      \end{subfigure}
    \caption{Total monthly sales given a department}
    \label{fig:SE13}
\end{figure}


\subsection{Prices-Focused Analysis}
The sell prices data has the weekly price for each product from \date(January 29, 2011) to \date(June 19, 2016), so the price data is for all the calender file weeks, unlike the sales data. The prices of products are not the same in different stores. We also found that the product with the greatest price change is HOUSEHOLD_2_406 sold in WI_2 store, having a minimum price of $3.26$ USD and a maximum price of $107.32$ USD. Moreover, There are $8247$ products, in different stores, that don't undergo any price change through the $5$ years; with $3497$ household items, $2907$ foods items, and $1843$ hobbies items. Some of these products' prices haven't changed in any of the store, like: FOODS_3_154, HOUSEHOLD_2_322, HOUSEHOLD_1_538, FOODS_3_309 and HOBBIES_2_014. Columns descriptions are in \Cref{tab:prices-desc}.

\begin{table}[H]
    \centering
    \caption{Prices details}
    \label{tab:prices-desc}
    \begin{tabular}{@{}lllll@{}}
    \toprule
    File            & Column name                                                                                                  & Column description &  &  \\ \midrule
    \multirow{4}{*}{sell\_prices.csv} & \texttt{store\_id}   & store ID                                                    \\
                              & \texttt{item\_id}    & the product ID coded as in \Cref{tab:sales-desc}                           \\
                              & \texttt{wm\_yr\_wk}  & code the week year from the dataset starting date           \\
                              & \texttt{sell\_price} & selling price of each product, for each week, in each store
    \\ \bottomrule
    \end{tabular}
    \end{table}

\subsubsection{Prices Exploration Figures}
\textbf{What are the most profitable states? and categories?} \\
    Comparing the total revenue with the total sales, we find that they match, with FOODS being the highest grossing category, and California being the most profitable state. This is shown in \Cref{fig:SE14}.
    \begin{figure}[H]
        \begin{subfigure}{.5\textwidth}
          \centering
          \includegraphics[width=.8\linewidth]{{"Revenue percentage of each category"}}  
          \caption{Revenue percentage of each category}
        \end{subfigure}
        \begin{subfigure}{.5\textwidth}
          \centering
          \includegraphics[width=.8\linewidth]{{"Revenue percentage of each state"}}  
          \caption{Revenue percentage of each state}
        \end{subfigure}
      \caption{Revenue percentage}
      \label{fig:SE14}
  \end{figure}

\textbf{What are the most profitable items?}\\
The top $5$ items that generate the most revenue are FOODS_3 items, and they are\footnote{in million USDs}:\begin{itemize}
    \item FOODS_3_120, sold in CA_3	with revenue $197541.66$.
    \item FOODS_3_090, sold in CA_3	with revenue $173741.55$.
    \item FOODS_3_586, sold in TX_2	with revenue $171385.16$.
    \item FOODS_3_120, sold in CA_1 with revenue $151387.02$.
    \item FOODS_3_586, sold in TX_3 with revenue $137892.80$.
\end{itemize}
  
\par\noindent\rule{\linewidth}{2px}
\begin{itemize}
    \item Haven't started: \begin{enumerate}
        \item total revenue of each category given the store
        \item total revenue in each month
        \item total revenue in each year
        \item total revenue in each store
    \end{enumerate}
    \item Doing:  \begin{enumerate}
    \item SNAP sales analysis
    \end{enumerate}
\end{itemize}

\clearpage
\section{Forecasting}

Interesting questions that our project should answer:
\begin{enumerate}
    \item Which department is the most important by sales? Which department is the least important?  \textcolor{red}{answered.}
    \item Which items are commonly bought together in each store? What about which items are bought together in each state? \textcolor{red}{We can't answer this since we don't have individual shopping data.}
    \item Which feast days generate the most sales? Which feast days generate the least? \textcolor{red}{answered first part.}
    \item Which items are the most profitable (i.e. best revenue*nsales)? Where are these sold? \textcolor{red}{answered.}
    \item Can we predict the item sales accurately?
\end{enumerate}

\end{document}

\clearpage
\section{Proposal}
\ahmed{Will probably remove this later.}
\subsection{Idea}
We want to forecast the unit sales of different products sold by Walmart markets in the United States. More precisely, given hierarchical sales data from Walmart we will forecast daily sales for two subsequent 28-day time periods.

\subsection{Planned Approach}
We aim to use a mix of statistical features specific to time series data combined with neural networks. For deseasonalization we can use techniques like exponential smoothing or other statistical procedures on sliding windows. We hope to understand the seasonality patterns in the data first and then use a recurrent neural network to predict the trends.


