\documentclass[12pt]{article}

\usepackage{underscore}
\usepackage{rka-style}

\usepackage[margin=0.875in]{geometry}
\usepackage{accents, hyperref, chngcntr, tikz-cd}
\usepackage[linesnumbered, noline]{algorithm2e}
\usepackage{natbib}
\bibliographystyle{plainnat}
\setcitestyle{authoryear,round,citesep={;},aysep={,},yysep={;}}

\hypersetup{ %
    pdfborder=0 0 0,
    pdfpagemode=UseNone,
    colorlinks=true,
    linkcolor=blue,
    citecolor=blue,
    filecolor=blue,
    urlcolor=blue,
    pdfview=FitH}

\graphicspath{ {figures/} }
\usepackage{cleveref,booktabs}

% TO DO NOTES
\usepackage[colorinlistoftodos,bordercolor=orange,backgroundcolor=orange!20,linecolor=orange,textsize=scriptsize]{todonotes}
\newcommand{\sra}[1]{\todo[inline]{{\textbf{Sara:} \emph{#1}}}}
\newcommand{\ahmed}[1]{\todo[inline]{{\textbf{Ahmed:} \emph{#1}}}}

% Preamble End
% Document Start

\begin{document}

\title{Big Data Project Report}
\author{Ahmed Essam, Ahmed Khaled, Sara Maher, Ibrahim Mahmoud}

\maketitle

\abstract{This report gives a time series analysis of historical Walmart sales data about more than three thousand different products sold across three different states in the United States. The data is part of the M5 forecasting competition \citep{Makridakis2020} and in addition to the time series analysis we also propose and describe a forecasting system and analyze its prediction capabilities.}

\tableofcontents
\clearpage
\section{Introduction}
Sales forecasting is important because bla bla

\subsection{Problem Definition}
We are given Walmart sales data from ten stores in three different states. Our project can be divided into two main problems: the first is to answer business-related questions on the dataset (data analysis) and the second is to devise a machine learning model that can predict the sales of each item offered by the stores twenty eight days into the future. 

This document is organized as follows: we review the dataset next, perform exploratory data analysis in the succeeding section, and outline our efforts at developing a forecasting model (and its evaluation) in the last section.

\subsection{Dataset Overview}
The dataset is given by the \emph{M5 competition} available on Kaggle\footnote{\url{https://www.kaggle.com/c/m5-forecasting-accuracy}.} which is the current installment in the popular M competition series on forecasting \citep{Makridakis2020}. The dataset includes time series data of the sales of various Walmart store products divided hierarchically by the item level, department, product category, and geographical area. The dataset also includes explanatory variables such as price, promotions, day of the week, and special events (e.g. Valentine's Day, Orthodox Easter, and the Super Bowl, one of the largest sporting events in American Football). There are $3,075$ products classified in $3$ product categories and $7$ product departments. The products are sold across $10$ stores in $3$ different states (California, Texas, and Wisconsin). The total number of M5 series across the entire hierarchy is $42,840$.  The dataset guide is given on the \href{https://mk0mcompetitiont8ake.kinstacdn.com/wp-content/uploads/2020/02/M5-Competitors-Guide_Final-1.pdf}{on the M5 competition website}.

\clearpage
\section{Data Analysis}
\subsection{Dataset Files}
The dataset is divided into four files, each of which is described in \Cref{tab:dataset-desc}.

\begin{table}[h]
    \centering
    \caption{Dataset description}
    \label{tab:dataset-desc}
    \begin{tabular}{@{}ccc@{}}
    \toprule
    File name                  & Description                                                                                                & Shape               \\ \midrule
    sales_train_validation.csv & Contains the sales data for each item                                                                      & $30490 \times 1919$ \\
    calendar.csv & \begin{tabular}[c]{@{}c@{}}Contains calendar dates as well as \\ any events that happen on each date.\end{tabular} & $1969 \times 14$ \\
    sell_prices.csv            & \begin{tabular}[c]{@{}c@{}}Contains the sell price for each item\\ divided by store and week.\end{tabular} & $6841121 \times 4$  \\
    sample_submission.csv      & \begin{tabular}[c]{@{}c@{}}Contains sample submission forecasting\\ for the competition.\end{tabular}      & $60980 \times 29$   \\ \bottomrule
    \end{tabular}
\end{table}

The columns in each file are described in Tables~[??].


\section{Exploratory Data Analysis}
The data consists of 3 files: \begin{itemize}
    \item \texttt{calender.csv}: containing the columns: \texttt{date}: the date in Year-Month-Day format, \texttt{wm\_yr\_wk}: code the week year from the dataset starting date, \texttt{weekday}: the weekday, \texttt{wday}: the weekday's number, starting from Saturday, \texttt{month}: the month, \texttt{year}: the year, \texttt{d}: the incremental ID of the day in the dataset, \texttt{event\_name\_1}: name of the event occurring in this day, if one exists, \texttt{event\_type\_1}: type of the event occurring in this day, if one exists, \texttt{event\_name\_2}: name of the second event occurring in that date,if one exists, \texttt{event\_type\_2}: type of the second event occurring in that date, if one exists, \texttt{snap\_CA}, \texttt{snap\_TX}, \texttt{snap\_WI}: 1 if SNAP\footnote{The Supplement Nutrition 
    Assistance Program (SNAP) provides low income families and individuals with an Electronic Benefits Transfer 
    debit card to purchase food products, this purchasing process is done monthly across 10 days} purchases are allowed on this date.
    \item \texttt{sales\_train\_validation.csv}: containing the columns: \texttt{id}, with the id of the item codes as \textit{item\_id}\_\textit{store\_id}\_validation, \texttt{item\_id}: product ID, \texttt{dept\_id}: department ID,
    \texttt{cat\_id}: category ID, \texttt{store\_id}: store ID, \texttt{state\_id}: state ID. and extra 1913 columns containing the total sold items each day from \texttt{d\_1} to \texttt{d\_1913}.
\end{itemize}
\subsection{Calender-Focused Analysis}
The data starts at \date{January 29, 2011} to \date{June 19, 2016}, with day-count up to 1969. The dataset contains 30 events, some of them are repeated (counting duplicates, there are 167 events). They are divided into: Cultural, National, Religious and Sporting. With Religious events being the most frequent with 55 occurrence. Checking the calender data, no missing values were found.
\subsection{Sales-Focused Analysis}
Checking the calender data, no missing values were found. But the sales data is for 1913 days only,unlike the calender. The data contains 30490 rows and 1919 columns. The data contains 3 states; California (CA), Texas (TX) and Wisconsin (WI), with 4, 3, and 3 stores respectively. It also contains 4 categories; Hobbies, Household, Foods, with 2, 2, and 3 departments respectively.
\subsubsection{Sales Exploration Figures}
    \begin{figure}[H]
	\centering
        \includegraphics[width=0.65\textwidth]{{"Share of each state from the products"}}
    \end{figure}
    Here, it is clear that CA has the more products, this could just be a result of it having more stores.
    \begin{figure}[H]
    \centering
        \includegraphics[width=0.65\textwidth]{{"Share of each store from the products"}}
    \end{figure}
    Investigating further, we find that this is just the case. Meaning that Walmart equalizes between these markets.
    \begin{figure}[H]
    \centering
        \includegraphics[width=0.65\textwidth]{{"Share of each category from the products"}}
    \end{figure}
    \begin{figure}[H]
    \centering
        \includegraphics[width=0.65\textwidth]{{"Share of each department from the products"}}
    \end{figure}
        Walmart stocks up on products of department FOODS\_3 the most.
    \begin{figure}[H]
        \centering
            \includegraphics[width=0.65\textwidth]{{"Sales percentage of each state"}}
    \end{figure}
    \begin{figure}[H]
        \centering
            \includegraphics[width=0.65\textwidth]{{"Sales percentage of each store"}}
    \end{figure}
    \begin{figure}[H]
        \centering
            \includegraphics[width=0.65\textwidth]{{"Sales percentage of each category"}}
    \end{figure}
    \begin{figure}[H]
        \centering
            \includegraphics[width=0.65\textwidth]{{"Sales percentage of each department"}}
    \end{figure}
    \begin{figure}[H]
        \centering
            \includegraphics[width=0.65\textwidth]{{"Total Sales per State for Each Category"}}
    \end{figure}
    \begin{figure}[H]
        \centering
            \includegraphics[width=0.65\textwidth]{{"Total Sales per category for each state"}}
    \end{figure}
    \begin{figure}[H]
        \centering
            \includegraphics[width=0.65\textwidth]{{"Total Sales per Store for Each Category"}}
    \end{figure}
    \begin{figure}[H]
        \centering
            \includegraphics[width=0.65\textwidth]{{"Total Sales per Category for Each Store"}}
    \end{figure}
    \begin{figure}[H]
        \centering
            \includegraphics[width=0.65\textwidth]{{"Total Sales per State for Each Department"}}
    \end{figure}
    \begin{figure}[H]
        \centering
            \includegraphics[width=0.65\textwidth]{{"Total Sales per Department for each State"}}
    \end{figure}
    \begin{figure}[H]
        \centering
            \includegraphics[width=0.65\textwidth]{{"Total Sales per Store for Each Department"}}
    \end{figure}
    The following 4 figures the total sales plot aren't the most informative, due to large variation between different departments, so they were normalized and stacked for better visualization:
    \begin{figure}[H]
        \centering
            \includegraphics[width=0.65\textwidth]{{"Percentage of Each Store Sales for Each Department"}}
    \end{figure}

    \begin{figure}[H]
        \centering
            \includegraphics[width=0.65\textwidth]{{"Total Sales per Department for Each Store"}}
    \end{figure}
    \begin{figure}[H]
        \centering
            \includegraphics[width=0.65\textwidth]{{"Percentage of Each Department Sales for Each Store"}}
    \end{figure}
    \begin{figure}[H]
        \centering
            \includegraphics[width=\textwidth]{{"Total sales over time"}}
    \end{figure}
    \begin{figure}[H]
        \centering
            \includegraphics[width=\textwidth]{{"2011 Q1"}}
    \end{figure}
    \begin{figure}[H]
        \centering
            \includegraphics[width=\textwidth]{{"2011 Q2"}}
    \end{figure}
    \begin{figure}[H]
        \centering
            \includegraphics[width=\textwidth]{{"2011 Q3"}}
    \end{figure}
    \begin{figure}[H]
        \centering
            \includegraphics[width=\textwidth]{{"2011 Q4"}}
    \end{figure}

\par\noindent\rule{\linewidth}{2px}
Draft of the analysis points we should cover, divided into 2 parts: doing, and haven't started, since the done ones can be seen above. Will be updated with more points, if we think of more, and the current state of each one. This is just a draft to do list and will surely be replaced in the final document.
\begin{itemize}
    \item Haven't started: \begin{enumerate}
        \item total revenue in each store
        \item total revenue in each state
        \item total revenue of each category
        \item total revenue of each category given the state
        \item total revenue of each category given the store
        \item total revenue of each category given an event happens in that day
        \item total revenue in each weekday
        \item total revenue in each month
        \item total revenue in each year
    \end{enumerate}
    \item Doing:  \begin{enumerate}
        \item total sold items of each category given an event happens in that day
       \item total sold items of each department given an event happens in that day
        \item total sold items in each weekday
        \item total sold items in each month
    \end{enumerate}
\end{itemize}

\clearpage
\section{Forecasting}

Interesting questions that our project should answer:
\begin{enumerate}
    \item Which department is the most important by sales? Which department is the least important?
    \item Which items are commonly bought together in each store? What about which items are bought together in each state?
    \item Which feast days generate the most sales? Which feast days generate the least?
    \item Which items are the most profitable (i.e. best revenue*nsales)? Where are these sold?
    \item Can we predict the item sales accurately?
\end{enumerate}

\end{document}

\clearpage
\section{Proposal}
\ahmed{Will probably remove this later.}
\subsection{Idea}
We want to forecast the unit sales of different products sold by Walmart markets in the United States. More precisely, given hierarchical sales data from Walmart we will forecast daily sales for two subsequent 28-day time periods.

\subsection{Planned Approach}
We aim to use a mix of statistical features specific to time series data combined with neural networks. For deseasonalization we can use techniques like exponential smoothing or other statistical procedures on sliding windows. We hope to understand the seasonality patterns in the data first and then use a recurrent neural network to predict the trends.


